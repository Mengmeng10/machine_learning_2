\section*{Task 1}

\subsection*{i}

\begin{gather*}
\epsilon(W) = \sum_{i} |\vec{X_{i}} - \sum_j W_{ij} \vec{X_{j}}| ^{2} \\
\tilde{\epsilon}(W) = \sum_{i} |\alpha \vec{X_{i}} - \sum_j W_{ij} \alpha \vec{X_{j}}| ^{2} \\
\tilde{\epsilon}(W) = \sum_{i} |\alpha (\vec{X_{i}} - \sum_j W_{ij} \vec{X_{j}})| ^{2} \\
\tilde{\epsilon}(W) = \sum_{i} \alpha^2 |(\vec{X_{i}} - \sum_j W_{ij} \vec{X_{j}})| ^{2} \\
\tilde{\epsilon}(W) = \alpha^2 \sum_{i} |(\vec{X_{i}} - \sum_j W_{ij} \vec{X_{j}})| ^{2} \\
\end{gather*}

Since $\alpha^2$ in $\tilde{\epsilon}(W)$ does not influence the weights of the maximum we get the same result as in ${\epsilon}(W)$

\subsection*{ii}

\begin{gather*}
\epsilon(W) = \sum_{i} |\vec{X_{i}} - \sum_j W_{ij} \vec{X_{j}}| ^{2} \\
\tilde{\epsilon}(W) = \sum_{i} |(\vec{X_{i}} + \vec{v}) - \sum_j W_{ij}  (\vec{X_{j}} + \vec{v})| ^{2} \\
\tilde{\epsilon}(W) = \sum_{i} |(\vec{X_{i}} + \vec{v}) - (\sum_j W_{ij} \vec{X_{j}}) - (\sum_j W_{ij} \vec{v}))| ^{2} \\
\tilde{\epsilon}(W) = \sum_{i} |(\vec{X_{i}} + \vec{v}) - (\sum_j W_{ij} \vec{X_{j}}) - \vec{v} (\sum_j W_{ij}))| ^{2} \\
\tilde{\epsilon}(W) = \sum_{i} |\vec{X_{i}} + \vec{v} - (\sum_j W_{ij} \vec{X_{j}}) - \vec{v}| ^{2} \\
\tilde{\epsilon}(W) = \sum_{i} |\vec{X_{i}} - (\sum_j W_{ij} \vec{X_{j}})| ^{2} = \epsilon(W)
\end{gather*}

\subsection*{iii}

\begin{gather*}
\epsilon(W) = \sum_{i} |\vec{X_{i}} - \sum_j W_{ij} \vec{X_{j}}| ^{2} \\
\tilde{\epsilon}(W) = \sum_{i} |U\vec{X_{i}} - \sum_j W_{ij} U\vec{X_{j}}| ^{2} \\
\tilde{\epsilon}(W) = \sum_{i} |U\vec{X_{i}} - U(\sum_j W_{ij} \vec{X_{j}})| ^{2} \\
\tilde{\epsilon}(W) = \sum_{i} |U(\vec{X_{i}} - \sum_j W_{ij} \vec{X_{j}})| ^{2} \\
\tilde{\epsilon}(W) = \sum_{i} |\vec{X_{i}} - \sum_j W_{ij} \vec{X_{j}}| ^{2} = \epsilon(W)
\end{gather*}
In the last step we used the fact that an orthogonal matrix does not change the length of a vector because
\begin{gather*}
|U\vec{v}|^2 = \langle U\vec{v}, U\vec{v}\rangle = \langle \vec{v}, U^T U\vec{v}\rangle = \langle \vec{v}, \vec{v}\rangle = |\vec{v}|^2
\end{gather*}

\textbf{General Rotations:} We get a rotation around arbitrary points by combining translation and rotation. If we want to rotate around point $A$, we can first do a translation so that point $A$ becomes the origin. Then we rotate around the origin and after that we reverse the translation.

\clearpage
